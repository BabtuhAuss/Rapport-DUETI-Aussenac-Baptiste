\subsection{Introduction}

Maintenant que nous avons définis notre vie au Québec, nous allons parlé la partie technique de notre année. En effet, nous avions pour but d'enrichir nos compétences techniques dans un autre environnement que le système français, et nous verrons que le résultat est intéressant.
\\
Lors de notre deuxième trimestre, nous avions une matière appelé "Stage-projet" permettant d'utiliser les notions apprises au cours des années précédents sur un projet que nous devions définir. En plus de cela, nous avions l'obligation d'être sous la tutelle d'un professeur souhaitant participer à notre projet.\\

Seul ou en équipe, cette matière est une occasion parfaite pour réaliser un projet qui nous intéresse, nous avons donc décidé de créer une équipe pour un seul projet commun.\\

Après avoir fait l'équipe, nous avons cherché un projet qui nous conviendrait tous, qui toucherait le plus de sujets possible, et qui, par ailleurs, nous plairait le plus !\\

Après quelques temps, nous sommes arrivé à cette conclusion : nous voulions travailler sur un gros projet avec différentes branches. C'est-à-dire, créer un serveur avec une nouvelle technologie pour certains, une interface client web inédite, un système de gestion de base de données, et tout ça dans un système de version de contrôle opérationnel (git). Et c'est ce que le projet suivant nous offre. \\

Voici donc l'intitulé de notre projet : \\

\noindent
{\rule{\linewidth}{0.5mm} }
\begin{center}
\color{red}
    \textbf{Application mobile de calendrier groupé avec une gestion de paiement et d'une messagerie instantanée, géré par Socket et un serveur REST.}
\end{center}
\noindent
{\rule{\linewidth}{0.5mm} }

Ce nom relativement long et complet a fait son effet auprès de notre tuteur ainsi qu'au département informatique. 
\\
À la suite de cela, nous avons demandé à notre professeur favori, \textbf{Eric Dallaire}, de nous accompagner sur le projet, car il était le mieux placé pour participer au projet. De part le fait qu'il nous a donné des cours de Bases de données, et d'informatique mobile, il était le candidat idéal.\\

Après avoir parlé avec lui du projet, avant la période de relâche en décembre; il a accepté d'être notre tuteur de projet et c'est ainsi que nous commencions notre projet officieusement. Nous entamions officiellement notre projet en janvier (début de la session d'hiver, en Janvier) avec une avance par rapport aux autres étudiants, ayant attendu le début de la session pour travailler. 

Et voici comment nous allons présenter notre projet dans ce rapport : 

\begin{itemize}
    \item Dans la partie \textbf{Fonctionnalités} nous verrons le contenu du projet, les opérations qu'un utilisateur de l'application peut faire avec quelques illustrations de notre travail.
    \item Dans la partie \textbf{Outils Utilisés}, Nous parlerons des technologies que nous avons mis en place pour développer notre projet, de la conception à la production et quelques captures de code pour comprendre les enjeux de chacun de ces outils.
    \item Enfin, c'est dans \textbf{Gestion du projet} nous observerons les différentes tâches que nous avons effectués pour gérer au mieux notre projet.
\end{itemize}

\subsection{Fonctionnalités}

Lorsque l'on regarde notre application de loin, il semble que c'est un agenda classique permettant de s'organiser entre les différents évènements et activités. Cependant, son intérêt est de gérer plusieurs calendriers par rapport à un groupe de personnes. liant ainsi les utilisateurs pour faciliter la gestion entre eux. Par exemple, les membres du club de bridge de Chicoutimi peuvent se rejoindre sur un calendrier unique afin de voir les différentes séances d'entraînement, tournois ou bien soirée club chez soi. En plus de cela, les membres du club de bridge peuvent communiquer entre eux via une messagerie instantanée qui est implémentée dans le groupe et enfin, pour chaque évènement, les membres du club de bridge peuvent ajouter des dépenses liées à l'activité et ainsi offrir un équilibre financier entre les participants d'un évènement, pour éviter tout conflit entre les différents membres du club.
\\

Un utilisateur est unique par son login et son adresse mail. il se connecte avec un mot de passe qu'il aura créé lors de son inscription dans la partie "register". De plus, lors de la création du compte, il obtient un calendrier personnel avec le code couleur de son choix. Cependant, s'il désire avoir d'autres calendriers (pour les partager avec d'autres utilisateurs) il peut en ajouter grâce à un titre, une description et un thème qui lui sera demandé par la suite.
Il à la possibilité de rejoindre un calendrier si un membre de ce dernier le lui partage à l'aide d'un lien qui lui serait envoyé. Après avoir entré le lien dans le champs de saisie approprié, le calendrier est ajouté à la liste des calendriers possédés par l'utilisateur et il peut ainsi naviguer entre les différents agendas. 
\\

L'usager souhaitant \textbf{enrichir un calendrier d'évènements} doit simplement se rendre vers le calendrier en question et remplir les zones d'informations tel que le titre, la date de début, la date de fin, incluant l'heure de début et de fin, et une brève description de l'évènement. Il sera ensuite ajouté sur la liste des évènements.
\\
Pendant la partie conception et analyse de notre besoin, nous devions trouver une fonctionnalité qui démarquerait notre agenda des multiples applications qui puissent exister sur le marché.
C'est grâce à notre mode de vie en collocation que nous avons eu l'idée d'ajouter un module de gestion des paiements en fonction d'un évènement. Cette fonctionnalité étant peu représenté dans des agendas, nous pensions qu'il était utile de pouvoir grouper dans une application, un moyen permettant d'observer les dépenses de chacun dans des évènements et de rendre plus facile les remboursements entre les participants

\newpage

\subsubsection{Structure de l'application}

Il est évident que l'application doit avoir une interface afin d'interagir avec un utilisateur. De plus, son rôle est principalement de traiter une grande proportion d'informations (agendas, évènements, comptes utilisateurs, etc, messages), il faut une SGBD avec de solides bases pour être le plus performant. avec Robert nous avons donc réfléchis aux différentes règles que notre structure doit réspecter.
Le plus dur étant que nous n'avions pas de cahier des charges, nous devions, par conséquent, définir les limites de l'application. Et c'est à la suite de plusieurs discussions entre nous que nous avons eu le résultat suivant :

\insererfigure{images/travail_recherche/premier_rendu_structure_bd.png}
{19cm}
{Premier rendu de la structure de la SGBD}
{Première idée de la bd}
{2cm}

Certes sur cette image, petite, nous ne voyons pas vraiment les objets mais elle est plus visible dans la partie \textbf{Annexes}, mais nous observons quand même que notre projet possède assez d'instances pour nous faire travailler.
\\
À la suite de cette première idée de structure de stockage d'informations, nous avons pu parler avec notre professeur tuteur et il nous donna un retour plutôt intéressant, enrichissant pour les étudiants qui souhaitait en savoir d'avantages sur l'architecture et la conception d'une BD.

\insererfigure{images/travail_recherche/photo_reunion_3.jpg}
{17cm}
{Retour du professeur de l'architecture de la SGBD}{
Retour du professeur SGBD}
{2cm}


La communication avec l'application mobile était le point le plus important du projet, car il fallait savoir toutes les données que nous envoyions et que nous devions restreindre.

Nous rédigions donc, à nous quatre, quelques cas d'utilisations que nous montrions au tuteur pendant la première réunion, comme nous pouvons le voir dans le brouillon ci-dessous.

\insererfigure{images/travail_recherche/photo_reunion_1.jpg}
{17cm}
{Élements pour communiquer entre l'application mobile et le serveur}{
Élements pour communiquer entre l'application mobile et le serveur}
{2cm}

\subsubsection{Interface de l'application}
Au début du projet, en parallèle de la conception de la SGBD, Victor et Enguéran ont pu s'occuper de la navigation au sein de l'application mobile, en fonction des scénarii possibles, et nous retenions en premier temps cette navigation possible.

\insererfigure{images/travail_recherche/photo_reunion_2.jpg}
{15cm}
{Retour du professeur après la réunion du premier livrable}{
Retour du professeur Navigation}
{2cm}

C'est sur ses bases, approuvées par le tuteur, que nous avons commencé à réaliser notre interface mobile \footnote{Notez qu'il est possible de voir les illustrations de notre application dans la partie annexe}.\\

Alors que j'administrais avec Enguéran le serveur et la base de données, Robert et Victor concrétisaient le développement de l'application mobile en produisant des fenêtres simples d'utilisation (User-Friendly). Une application mobile ne doit pas être dotée de milliers de boutons, il fallait donc réfléchir aux éléments présents dans chaque fenêtre de manière à ce que l'utilisateur ne se perde pas, et puisse apprécier l'expérience de ce système d'agendas.

\subsubsection{Mécanisme du serveur et de l'application mobile}

Le serveur que nous avons créé était un serveur de REST qui récupérait des requêtes HTTP et qui traitait ensuite les différentes informations qu'un téléphone connecté à internet lui envoyait. Le calcul, et la récupérations des données de la SGBD sont donc fait en arrière plan, de façon distribuée, comme nous l'avions appris en deuxième année de DUT ainsi qu'au premier semestre de l'UQAC. En effet, notre serveur lançait deux applications, Socket et REST, l'un s'opérant pour un canal constant pour accéder aux messages, l'autre pour récupérer de grandes données tel que la liste de tout les événements des calendriers d'un utilisateur.\\

L'application mobile, envoie une multitude de requêtes au serveur. Cette opération permet au téléphone d'afficher uniquement les informations reçues, ce qui le rend fluide et rapide avec son design épuré.


\subsection{Outils Utilisés}
Côté Mobile
-Ionic, Angular = Typescript, technologie nodeJS, HTML, CSS
-Génération d'une application web sur mobile
-Tests des fonctionnalités effectuées sur téléphone android, ios, ainsi que sur le navigateur web des pc

Côté serveur
python, REST, cx\_oracle (libraire pour communiquer avec la SGBD)

Côté SGBD : oracle et sqlmodeller pour la conception (en plus de papier crayon)
Gestion de la base de données avec un script de réinitialisation

Git pour le système de version de projet

Editeur de texte Visual studio code pour comprendre le principe du lancement des programmes (sans ide)

\subsection{Gestion du projet}
2 gits

Kanban avec to do comme trello

Parler pdt la journée des choses à faire

Communication avec le tuteure tout les mois de façon agile en donnant un livrable

Travail sur visual studio code avec le live share : module permettant de coder à plusieurs sur un même fichier, utile pour faire du pair programming qui était une méthode efficace

Baptiste s'occupant de la partie SBGD + serveur ainsi que support pour Ionic,
Enguérant traitant les serveurs, ajoutant des fonctionnalités comme le paiement et un algorithme de réduction de transferts d'argent,
Victor et Robert travaillant Full Stack sur le serveur REST et sur l'interface mobile, et la communication entre le télephone et les serveurs

debuggage et fin d

\subsection{Idées d'implémentation}

Bien sûr, un projet de cette ampleur est ambitieux, une multitude de chemins s'offre à nous pour ce projet.

\subsection{S'adapter avec le virus}

Comme tout les étudiants en secondaire, nous avons du travailler sur le projet pendant la crise sanitaire. 

debbugage et finalisation du projet