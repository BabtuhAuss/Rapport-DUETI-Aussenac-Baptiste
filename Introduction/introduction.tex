Lors de mon entrée à l'IUT, mon chemin était déjà tracé : alternance, puis école d'ingénieur. Tout ça était prévu, jusqu'à ce que je découvre, lors de ma deuxième année, une aubaine à prendre pour écrire un nouvel avenir tout aussi passionnant, celui dans l'Université du Québec à Chicoutimi (UQAC). Cette ouverture à l'internationnal, loin de la zone de confort que j'ai pu construire, était une experience à ne pas laisser passer. C'est pour cela que je suis parti un an au Québec.
Par ailleurs, l'idée était de se rapprocher des collègues de l'iut, collocataires et amis, dans la gestion de la vie.

\subsection{Objectifs professionnels}

L'objectif de partir à l'UQAC était de faire une double diplomation en ayant le DUETI en France ainsi que le Baccalauréat en Informatique au Canada (qui équivaut à une licence ici). De plus le diplôme canadien permet une continuité du cursus dans le pays. En parallèle, nous avons pu découvrir l'enseignement au sein d'une université nord-américaine, différent du système français.

\subsection{Objectifs personnels}

Pendant cette année, nous avons aussi définis une multitude d'objectifs personnels.
Premièrement, découvrir une nouvelle société et une nouvelle culture ainsi qu'un environnement variant beaucoup de la France (particulièrement en hiver). 
Deuxièmement, nous souhaitions acquérir plus d'autonomie et d'organisation en nous retrouvant dans une situation que nous n'avions jamais connu sur le long terme.
Enfin, cette année était l'occasion

\subsection{Présentation du rapport}

Ce rapport est composé de deux partie:\newline
Une première partie permettra de décrire précisément l’environnement dans lequel nous avons vécu notre année à l’UQAC; à savoir l’État du Québec, la région de Saguenay-Lac-St-Jean, l’université, et un approfondissement personnel de chacun.\newline
La seconde partie détaillera le travail que j’ai effectué avec mes collègues de l'iut avec qui je suis parti. Nous y
développerons autant la partie technique du travail que la partie management de projet.

