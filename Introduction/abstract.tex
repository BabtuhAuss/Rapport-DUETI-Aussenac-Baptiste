When we begun the IUT, We tought that our paths were already mapped out: apprenticeship, then engineering school for the most part. All of this was planned, until we discovered, during our second year of the IUT, a new opportunity to pursue our studies at the Université du Québec à Chicoutimi (UQAC). The opening to the international scene, far from the comfort zone we built, has to be an experience which we can't avoid. That's why we've taken this chance and spent a year in Quebec. Moreover, the idea was to have a good relationship with the colleagues of the IUT, because, we meant to live in the same flat.

\subsection{Professional issues}

The puropse of going to UQAC was to do a double degree by having the DUETI in France and the Baccalauréat en Informatique in Canada (which is equivalent to a bachelor's degree here). In addition, the Canadian diploma allows us, if we want, to continue our studies in the country. At the same time, we were able to discover teaching at a North American university.

\subsection{Personal issues}

During this year, we had also several personals goals.
Firstly, to discover a new society and a new culture as well as an environment that varies a lot from France (especially in winter). 
Secondly, we wanted to acquire more autonomy and organization by finding ourselves in a situation that we had never experienced in the long term.
Finally, we wanted to practice sport, as a team, to get closer while developping our body.

\subsection{Summary of the report}

This document consists of two parts:\newline
The first part will describe the environment in which we spent our year at UQAC: the State of Quebec, the Saguenay-Lac-St-Jean region, the university.\newline
The second part will detail the project we had in our second half of the year. We will develop both the technical part of the work and the project management part.

