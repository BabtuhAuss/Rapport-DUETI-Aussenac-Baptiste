Lors de notre entrée à l'IUT, nos chemins était déjà tracés : alternance, puis école d'ingénieur pour la plupart. Tout ça était prévu, jusqu'à la découverte, lors de la deuxième année d'IUT, une aubaine à prendre pour écrire un nouvel avenir tout aussi passionnant, celui dans l'Université du Québec à Chicoutimi (UQAC). Cette ouverture à l'international, loin de la zone de confort que nous avions pu construire, était une expérience à ne pas laisser passer. C'est pour cela que nous sommes partis un an au Québec.
Par ailleurs, l'idée était de se rapprocher des collègues de l'iut, colocataires et amis, dans la gestion de la vie.\\

Nous tenons à remercier les professeurs Mme Barbot, Mme Fancett, pour nous avoir montré ce choix qui n'a déçu personne ici, Mme Ronsse pour nous avoir aidé dans l'administration au début de l'année, tout le corps enseignant de l'IUT de Vélizy qui nous ont permis d'avoir de merveilleux résultats grâce à leurs cours, et bien entendu le corps enseignant de l'UQAC (Eric Dallaire en particulier) sans qui nous ne nous serions pas amusé autant cette année.

Baptiste : Je tiens à remercier Antoine Berthier de m'avoir montré l'université, qui m'a grandement inspiré pour la suite de mes études !

\subsection{Objectifs professionnels}

L'objectif de partir à l'UQAC était de faire une double diplomation en ayant le DUETI en France ainsi que le Baccalauréat en Informatique au Canada (qui équivaut à une licence ici). De plus le diplôme canadien permet une continuité du cursus dans le pays. En parallèle, nous avons pu découvrir l'enseignement au sein d'une université nord-américaine, différent du système français.

\subsection{Objectifs personnels}

Pendant cette année, nous avons aussi définis une multitude d'objectifs personnels.
Premièrement, découvrir une nouvelle société et une nouvelle culture ainsi qu'un environnement variant beaucoup de la France (particulièrement en hiver). 
Deuxièmement, nous souhaitions acquérir plus d'autonomie et d'organisation en nous retrouvant dans une situation que nous n'avions jamais connu sur le long terme.
Enfin, nous voulions pratiquer du sport, d'équipe, pour nous rapprocher tout en évoluant notre corps

\subsection{Présentation du rapport}

Ce rapport est composé de deux partie:\newline
Une première partie permettra de décrire précisément l’environnement dans lequel nous avons vécu notre année à l’UQAC; à savoir l’État du Québec, la région de Saguenay-Lac-St-Jean, l’université, et un approfondissement personnel de chacun.\newline
La seconde partie détaillera le projet que nous avons eu au cours de notre deuxième semestre. Nous y développerons autant la partie technique du travail que la partie management de projet.

