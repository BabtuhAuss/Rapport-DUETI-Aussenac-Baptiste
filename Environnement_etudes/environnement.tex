\lipsum[8-2] %Effacer cette ligne et écrire le texte souhaité

\subsection{Description du Québec}

\subsubsection{Un mode de vie "A l'américaine"}

L’une des premières choses que nous avons faites en arrivant était du repérage, nous sommes arrivés sur un site avec de nombreux bâtiments reliés les uns aux autres par des passerelles, une salle de sport, un terrain de football américain, un stade d’athlétisme, beaucoup d’installations sportives, cela nous a ravis et nous pensions déjà aux corps de rêves que nous allions obtenir en plus des connaissances en informatiques et des futures activités qui auront lieu. 

Le premier jour on nous a annoncé la présence de casier, on avait déjà l’image des casiers vraiment personnalisés et propres à chaque personne, cependant nous avons rapidement compris que ces casiers étaient uniquement présents pour éviter de se balader avec les énormes manteaux d’hiver dans tout le bâtiment. 

La première Partie Universitaire mais pas la dernière, malgré le fait que l'on ne soit pas resté longtemps, nous avons pu constater le nombre de personne important sur le site de l’UQAC, avec 6000 élèves présents pendant cette année, il était compliqué d’accéder aux saucisses et au chamallow avec des files d’attente assez conséquentes. 

Grâce à la quantité de Partie Universitaire et notre intérêt dans la vie étudiante en réalisant du bénévolat lors de ces PU, nous avons rapidement créer des relations avec de nombreuses personnes appartenant aux divers associations et comités. 


Qui dit Amérique dit pancake et bacon, nous n’avons pas attendu longtemps avant d’adopter ce style de vie composé de petit déjeuner assez chargé : bacon, pancake, sirop d’érable, ils étaient nécessaires avec les séances de sport qui nous attendais dans la journée, ces petits déjeuner par la disparité entre les horaires de chacun était le seul repas que nous ne faisions pas tous ensemble.  

\subsubsection{Le Québec en hiver}

Combien de temps mettions nous pour aller sur le lieu de travail. 15 minutes à pied PTDRRR apres bus tu coco 

DE LA NEIGEEEEEEE 

DES TRACTEURS QUI DENEIGEEEEE 

DES TEMPETESSSSSSSSS 

\subsection{La région de Saguenay-Lac-St-Jean : recul de la ville et nature}

\subsection{L'Université du Québec à Chicoutimi}
\subsubsection{Ambiance}

Pas de stress 

Toujours à l’aise avec ces ostits de québécois 

Beaucoup de sport, donc toujours décontractés 

Beaucoup de temps libre 

Des français rencontrés un peu partout, souvent des bonnes surprises 

\subsubsection{Cours}

Premier semestre : 

BD 

Gestion de projet 

SD (MDRRRRRRRRRRRRRRRRRRRRRRR) S quoi ?

Sécu 

Info théorique 

Deuxième semestre : 

Projet 

Modélisation  

Mobile 

Algo 

Matière séparée : anglais/forage 

\subsubsection{Service aux Etudiants}

ROBERT je te laisse faire je pense on mets festival étudiant c'est suffisant 

\subsection{La vie dans Chicoutimi : l'histoire d'une collocation et d'une vie hors campus}

\subsubsection{Avis des étudiants : ROBERT LE DROLE ET LE STOCK}
\subsubsection{Avis des étudiants : ENGUERAN LE COACH ET LE MENTALISTE}

Une année incroyable de laquelle on ressort grandi, aussi bien physiquement que mentalement, pour ma part mon objectif était presque accompli, mes poulains sont désormais monstrueux physiquement, beaucoup de connaissances acquises, plus de projet en autonomie, plus d’interaction entre élèves, ici colocataires logiquement puisqu’on se connaissait déjà, des jolies demoiselles rencontrées, beaucoup de fous rires due à l’accent des québécois et à leur attrait pour des choses assez étranges (délire d’impro). 

\subsubsection{Avis des étudiants : BAPTISTE LE NOUVEAU ET LE CERVEAU}
j'aime
\subsubsection{Avis des étudiants : VICTOR LE PAPA ET L'ADMINISTRATEUR}


